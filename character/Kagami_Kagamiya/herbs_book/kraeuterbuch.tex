\documentclass[12pt,a4paper]{article}
\usepackage[utf8]{inputenc}
\usepackage[T1]{fontenc}
\usepackage[ngerman]{babel}
\usepackage{graphicx}
\usepackage{geometry}
\usepackage{fancyhdr}
\usepackage{titlesec}
\usepackage{parskip}
\usepackage{hyperref}
\usepackage{multicol}
\usepackage{xcolor}

\geometry{margin=2.5cm}
\pagestyle{fancy}
\fancyhf{}
\lhead{Herbarium Medicum}
\rhead{\thepage}

\titleformat{\section}{\large\bfseries\color{darkgreen}}{\thesection}{1em}{}
\definecolor{darkgreen}{rgb}{0.0, 0.4, 0.0}

\title{\Huge\bfseries Herbarium Medicum\\[0.2cm]\large Das persönliche Heilkräuterbuch von Kagami Kagamiya}
\author{}
\date{}

\begin{document}

\maketitle
\thispagestyle{empty}
\newpage

\tableofcontents
\newpage

% Beispiel-Eintrag: Schafgarbe
\section{Schafgarbe (Achillea millefolium)}

\begin{figure}[h!]
    \centering
    \includegraphics[width=0.5\textwidth]{schafgarbe.jpg}
    \caption*{Abbildung: Schafgarbe (selbst gezeichnet / eingeheftet)}
\end{figure}

\subsection*{Fundort}
Wiesen, Wegränder, lichte Wälder, Hochländer

\subsection*{Anwendung}
Wundheilung, blutstillend, fiebersenkend, krampflösend

\subsection*{Zubereitung}
\textcolor{gray}{(Hier eigene Rezepte oder Notizen eintragen)}

\subsection*{Forschung/Beobachtung}
\textcolor{gray}{(Hier eigene Forschungsergebnisse, Experimente oder Beobachtungen einfügen)}

\newpage

% Nächster Platzhalter
\section{[Krautname] ([Lateinischer Name])}

\begin{figure}[h!]
    \centering
    \includegraphics[width=0.5\textwidth]{bildname.jpg}
    \caption*{Abbildung: [Krautname]}
\end{figure}

\subsection*{Fundort}
\textcolor{gray}{Hier Fundort notieren...}

\subsection*{Anwendung}
\textcolor{gray}{Hier medizinische Wirkung oder Anwendung beschreiben...}

\subsection*{Zubereitung}
\textcolor{gray}{Zubereitungsform: Tee, Tinktur, Umschlag, etc.}

\subsection*{Forschung/Beobachtung}
\textcolor{gray}{Eigene Forschung, Notizen, Experimente...}

\newpage

% Weitere Abschnitte folgen...

\end{document}
